% Intended LaTeX compiler: xelatex
\documentclass[aspectratio=149,11pt,italian]{beamer}
\usepackage{graphicx}
\usepackage{longtable}
\usepackage{wrapfig}
\usepackage{rotating}
\usepackage[normalem]{ulem}
\usepackage{amsmath}
\usepackage{amssymb}
\usepackage{capt-of}
\usepackage{hyperref}
\institute{Università di Siena}
\usepackage{localheader}
\usepackage{tikz}
\usepackage{booktabs,tabularx,tabularray}
\usepackage{setspace}
\usepackage{quoting}
\usepackage[italian]{babel}
\usepackage{fancybox}
\usepackage{tabularray}
\usetheme{default}
\author{Massimo D'Antoni}
\date{2023-2024}
\title{Politica di bilancio\newline e debito pubblico}
\subtitle{Scienza delle Finanze}
\hypersetup{
 pdfauthor={Massimo D'Antoni},
 pdftitle={Politica di bilancio e debito pubblico},
 pdflang={Italian}}
\begin{document}

\maketitle

\section{Il deficit di bilancio}

%%%%%%%%%%%%%%%%%%%%%%%%%%%%%%%%%%%%%%%%%%%%
\begin{frame}{I saldi di bilancio}
\begin{center}
\centering
\includegraphics[width=12cm]{./figure/saldi-di-bilancio.pdf}
\end{center}
\end{frame}

%%%%%%%%%%%%%%%%%%%%%%%%%%%%%%%%%%%%%%%%%%%%
\begin{frame}{L'indebitamento netto nei principali paesi UE}
\begin{center}
\centering
\includegraphics[width=\textwidth]{./figure/deficit-4countries-2000-2022-color.pdf}
\end{center}
\end{frame}


%%%%%%%%%%%%%%%%%%%%%%%%%%%%%%%%%%%%%%%%%%%%
\begin{frame}{È giustificabile la spesa in deficit?}
\begin{itemize}
\item Spese in eccesso sulle entrate correnti si possono giustificare:
\begin{itemize}
\item con la necessità di effettuare \alert{investimenti} finalizzati ad aumentare lo
stock di capitale pubblico (es. infrastrutture);
\item con l'esigenza di stabilizzare l'economia in presenza di fluttuazioni
cicliche o eventi eccezionali (es. terremoti, alluvioni, guerre, pandemie),
anche al fine di evitare che la capacità produttiva possa essere compromessa
\end{itemize}
\item In questi casi le imposte correnti potrebbero essere insufficienti a
fronteggiare le necessità.
\item Il deficit può essere un modo per rilanciare la domanda in un'ottica keynesiana.
\item Può essere «giusto» sincronizzare il pagamento delle imposte e il godimento dei
benefici di investimenti che hanno effetti di lungo periodo.
\item In presenza di necessità di spesa concentrate nel tempo, ottimale distribuirne
il peso in termini di riduzione dei consumi, in modo più uniforme nel tempo.
\item \alert{Tuttavia}, la possibilità di rinviare nel tempo il pagamento delle imposte
può avere effetti deresponsabilizzanti sui governi.
\end{itemize}
\end{frame}

%%%%%%%%%%%%%%%%%%%%%%%%%%%%%%%%%%%%%%%%%%%%
\begin{frame}{Il dibattito sulla natura del debito pubblico}
L'analogia con il debito di un privato può essere fuorviante per una pluralità
di ragioni:
\begin{enumerate}
\item lo Stato, che ha vita virtualmente infinita, non ha necessità di restituire
il debito, può continuare a rinnovarlo (\emph{rollover});
\item lo Stato può, in caso di necessità, ripagare il debito creando moneta
\begin{itemize}
\item controindicazioni: inflazione, perdita di credibilità
\end{itemize}
\item spesso lo Stato si indebita con i propri cittadini, che sono ad un tempo
creditori (in quanto possessori di titoli di Stato) e debitori (in quanto
contribuenti);
\item il debito non è «pagato» dalle generazioni future: le risorse per
realizzare la spesa finanziata a debito sono distolte da quelle disponibili
oggi
\begin{itemize}
\item «Non possiamo combattere le battaglie di oggi con i cavalli di domani»
\item È un debito che «la mano destra deve alla mano sinistra»
\end{itemize}
\item C'è differenza tra debito interno e debito estero
\end{enumerate}
\end{frame}

%%%%%%%%%%%%%%%%%%%%%%%%%%%%%%%%%%%%%%%%%%%%
\begin{frame}{Il dibattito sulla natura del debito pubblico /2}
\begin{quoting}
\footnotesize
«Se si costruisce una ferrovia dal costo di 100 milioni, forsechè il terreno
sarà stato spianato, i terrapieni innalzati, i ponti costruiti, le gallerie forate, le stazioni erette, i binari lanciati con lavoro e con materiale futuro?
Mai no. Che cosa è il costo della ferrovia, se non la fatica durata nello
spianar terreni, innalzar terrapieni, forar gallerie, costruire ponti, fabbricare traversine rotaie locomotive carrozze e carri? Chi durò quella fatica?
I posteri od i viventi? […] Non esiste nessun mezzo per far sostenere ai
posteri il costo, la fatica, il dolore di nessuna spesa presente. Se noi vivi vogliamo fare una spesa dobbiamo pagarcela noi con i mezzi presenti,
dobbiamo volgere a quello scopo i mezzi che sarebbero disponibili per
raggiungere altri fini presenti.»

(L. Einaudi, 1940)
\end{quoting}
\end{frame}

%%%%%%%%%%%%%%%%%%%%%%%%%%%%%%%%%%%%%%%%%%%%
\begin{frame}{Il dibattito sulla natura del debito pubblico /3}
Tuttavia, la presenza di un debito rappresenta un costo:
\begin{itemize}
\item raccogliere le imposte per pagare gli interessi vincola la politica
economica e impone un costo all'economia in futuro;
\item il ricorso al debito potrebbe modificare le scelte di risparmio e
accumulazione di capitale.
\item Il debito pubblico rappresenta ricchezza per i privati?
\begin{itemize}
\item A un estremo potremmo rispondere di no, se i privati anticipano
correttamente il fatto che dovranno finanziare il debito con le proprie
imposte (equivalenza ricardiana).
\item Se manca tale capacità di anticipazione, l'emissione di debito potrebbe
ridurre il risparmio privato e determinare una riduzione della dotazione
di capitale privato.
\item In ottica keynesiana, al contrario, la spesa a debito potrebbe stimolare
l'economia e indurre maggiori investimenti privati.
\end{itemize}
\end{itemize}
\end{frame}

%%%%%%%%%%%%%%%%%%%%%%%%%%%%%%%%%%%%%%%%%%%%
\begin{frame}{L'equivalenza ricardiana}
\begin{itemize}
\item Il debito pubblico rappresenta ricchezza per i privati? L'argomento
tradizionale di Ricardo è che il finanziamento a debito equivale al
finanziamento con un'imposta sul patrimonio corrente (vedi
\emph{capitalizzazione} dell'imposta).
\item Barro (1974) ha ripreso questo argomento: individui razionali, non soggetti
a illusione finanziaria, anticipano il fatto che l'emissione del debito
comporterà in futuro il pagamento di imposte.
\item Il debito pubblico non è percepito come ricchezza in quanto interamente
compensato dal valore attuale delle imposte future.
\item Gli individui rispondono all'emissione di debito riducendo i propri consumi
(aumentando il risparmio). Ciò vale anche quando le imposte saranno pagate
dai discendenti, la cui utilità è «internalizzata» dalla generazione
presente.
\item L'argomento è stato utilizzato per contestare che la tesi keynesiana per cui
spesa in deficit ha effetti espansivi sull'economia.
\end{itemize}
\end{frame}

%%%%%%%%%%%%%%%%%%%%%%%%%%%%%%%%%%%%%%%%%%%%
\begin{frame}{Il debito pubblico nella contabilità nazionale}
Ai fini della contabilità nazionale il debito pubblico rappresenta un
sottoinsieme delle passività della Pubblica amministrazione. È rappresentato
da:
\begin{enumerate}
\item biglietti monete e depositi (sono incluse le monete metalliche, non le
banconote emesse dalla Banca centrale);
\item titoli diversi da azioni e derivati;
\item prestiti.
\end{enumerate}
Dal debito pubblico sono esclusi:
\begin{itemize}
\item debiti delle imprese pubbliche al di fuori del perimetro della P.A. (sono
invece inclusi i titoli di Stato nel bilancio di tali imprese);
\item i «crediti commerciali» della P.A. verso i fornitori;
\item le garanzie esplicite o implicite fornite dallo Stato;
\item il debito pensionistico.
\end{itemize}
Infine, il debito si considera comunemente al \emph{lordo} di eventuali attività.
\end{frame}

%%%%%%%%%%%%%%%%%%%%%%%%%%%%%%%%%%%%%%%%%%%%
\begin{frame}{Il debito pubblico in alcuni paesi europei}
\begin{center}
\centering
\includegraphics[width=.9\textwidth]{./figure/debito-eu-color.pdf}
\end{center}
\end{frame}

\section{L'aritmetica del debito pubblico}

%%%%%%%%%%%%%%%%%%%%%%%%%%%%%%%%%%%%%%%%%%%%
\begin{frame}{L'evoluzione del debito pubblico in funzione di deficit e crescita}
\begin{itemize}
\item Se trascuriamo la possibilità di finanziare il deficit aumentando il debito
monetario (\emph{signoraggio}), vale la relazione:
\begin{equation*}
\Delta B_{t}=B_{t}-B_{t-1}= D_{t}
\end{equation*}
dove $B_{t}$ è lo stock di debito al termine del periodo $t$, $D_{t}$ è il
deficit nel periodo $t$
\item dividendo per $Y_{t}$ (tutte le variabili in \% del PIL) abbiamo
\begin{equation*}
\frac{B_{t}}{Y_{t}}=\frac{B_{t-1}}{Y_{t-1}(1+n)}+\frac{D_{t}}{Y_{t}}
\quad\implies\quad
 b_{t}=\frac{b_{t-1}}{1+n}+d_{t}
\end{equation*}
dove $b_t=B_{t}/Y_{t}$ e $d_{t}=D_{t}/Y_{t}$. Ricordiamo che
$Y_{t}=Y_{t-1}(1+n)$.
\item Fissando $n$ e $d$, la dinamica di $Y_{t}/B_{t}$ può essere illustrata con
l'aiuto di un \href{https://docs.google.com/spreadsheets/d/1taAlA7ksKNweRqOguMfEDdVAYOXU3TlYrjClW6MumaE/edit?usp=sharing}{→foglio elettronico}: osserviamo che al crescere di $t$ il
rapporto converge ad un valore che dipende da $d$ e da $n$
\end{itemize}
\end{frame}
%%%%%%%%%%%%%%%%%%%%%%%%%%%%%%%%%%%%%%%%%%%%
\begin{frame}{L'aritmetica del debito: il rapporto debito/PIL in funzione di $d$ e $n$}
\begin{columns}
\begin{column}{.5\columnwidth}
\begin{itemize}
\item Fissando un livello di deficit e mantenendolo costante ($d_{t}=d$), abbiamo
la dinamica descritta in figura
\item il valore di equilibrio si ottiene ponendo $b_{t}=b_{t-1}=b^*$ nella
\begin{equation*}
  b_{t}=\frac{b_{t-1}}{1+n}+d_{t}
\end{equation*}
ovvero:
\begin{equation*}
  b^*=\frac{1+n}{n}\cdot d \approx \frac{d}{n}
\end{equation*}
\item l'equilibrio è stabile se $n>0$
\end{itemize}
\end{column}

\begin{column}{.5\columnwidth}
\begin{center}
\centering
\includegraphics[width=\textwidth]{./figure/debito-pubblico-sost-1.pdf}
\end{center}
\end{column}
\end{columns}
\begin{block}{}
I parametri di Maastricht ($b=60\%$ e $d=3\%$) sono coerenti se $n=5\%$,
ovvero, visto che $n\approx g + \pi$, con crescita reale $g=3\%$ e inflazione
$\pi=2\%$
\end{block}
\end{frame}

%%%%%%%%%%%%%%%%%%%%%%%%%%%%%%%%%%%%%%%%%%%%
\begin{frame}{L'aritmetica del debito: l'avanzo primario che stabilizza il debito}
\begin{itemize}
\item Scomponiamo il deficit in spesa per interessi e \alert{avanzo primario}:
\begin{equation*}
 D_{t}=iB_{t-1}-(T_{t}-G_{t}) 
\quad\implies\quad 
d_{t}=\frac{ib_{t-1}}{1+n}-a_{t}
\end{equation*}
dove $a_{t}=(T_{t}-G_{t})/Y_{t}$ è l'avanzo primario in percentuale del PIL.
\item Abbiamo dunque
\begin{equation*}
   b_{t} = \frac{b_{t-1}}{1+n} + \frac{ib_{t-1}}{1+n} - a_{t}.
\end{equation*}
\item Fissando $b_{t}=b_{t-1}$, possiamo calcolare \alert{il livello di $a_{t}$
necessario a stabilizzare il debito}. Abbiamo:
\begin{equation*}
   a_{t} = \left(\frac{i-n}{1+n}\right)b_{t-1} \; \approx (i-n)b_{t-1}
\end{equation*}
\begin{itemize}
\item se $n=0$ (crescita zero), deve essere $a_{t}=ib_{t-1}$
\item $\frac{i-n}{1+n}$ è spesso indicato come \emph{tasso di interesse aggiustato per
la crescita}
\item se $a_t$ è maggiore (minore) di $\frac{i-n}{1+n}b_{t-1}$ il livello di $b$
decresce (cresce).
\item l'avanzo primario necessario per stabilizzare il debito è tanto maggiore
quanto più alto è $b_{t-1}$, quanto più alto è $i$ e quanto più bassa è la
crescita $n$
\end{itemize}
\end{itemize}
\end{frame}

%%%%%%%%%%%%%%%%%%%%%%%%%%%%%%%%%%%%%%%%%%%%
\begin{frame}{Le determinanti della crescita del debito}
\begin{itemize}
\item Possiamo identificare le determinanti della crescita di $b$ scrivendo
\begin{equation*}
 b_{t}-b_{t-1}=\frac{i}{1+n}b_{t-1}-\frac{n}{1+n}b_{t-1}+a_{t}
 \;\;\approx (i-n)b_{t-1}-a_{t}
\end{equation*}
per cui la variazione di $b$ dipende positivamente dal tasso di interesse
$i$ e negativamente dal tasso di crescita $n$ e dall'avanzo primario $a$
\item In sintesi, la dinamica del debito dipende da tre variabili: l'avanzo
primario $a_t$, il tasso di interesse $i$ e il tasso di crescita $n$.
\item Notiamo che:
\begin{itemize}
\item quando $i>n$ è necessario un saldo primario $a_t$ positivo per
stabilizzare il debito
\item quando $n>i$ la dinamica del debito è stabile, è sufficiente fissare $a_t$
per avere convergenza.
\end{itemize}
\item Un problema è dato dal fatto che $a$ può influenzare sia $n$ che $i$.
\end{itemize}
\end{frame}

%%%%%%%%%%%%%%%%%%%%%%%%%%%%%%%%%%%%%%%%%%%%
\begin{frame}{L'evoluzione del debito italiano e le sue determinanti}
\begin{center}
\centering
\includegraphics[width=.9\textwidth]{./figure/crescita-debito-1980-2024-scomposizione-nominale-color.pdf}
\end{center}
\end{frame}

%%%%%%%%%%%%%%%%%%%%%%%%%%%%%%%%%%%%%%%%%%%%
\begin{frame}{L'evoluzione del debito italiano e le sue determinanti /2}
Alcune osservazioni:
\begin{itemize}
\item L'attuale situazione di debito elevato si è creata principalmente nel corso
degli anni '80. Il rapporto debito/PIL si è ridotto a partire da metà anni
'90 ed è tornato a crescere con la crisi
\item Le fasi di crescita rapida del debito sembrano dipendere più dall'andamento
della componente $(i-n)b_{t-1}$ che dall'avanzo primario, che è rimasto positivo
dopo il 1992
\begin{itemize}
\item Negli anni '70 deficit primari anche elevati, ma l'elevata crescita
nominale (superiore al tasso di interesse) evitava un aumento rapido del
debito
\item Negli anni '80 rallentamento della crescita nominale, elevati tassi di
interesse (liberalizzazione dei mercati dei capitali e politica monetaria
USA) e deficit in calo solo alla fine del periodo
\item Nei primi anni '90 consistente consolidamento fiscale
\item A fine anni '90 riduzione dei tassi di interesse per effetto della
stabilizzazione del cambio e dell'ingresso nell'euro
\item Con la crisi l'effetto è principalmente quello della crescita bassa o
negativa, nonostante lo sforzo di consolidamento
\end{itemize}
\end{itemize}
\end{frame}

%%%%%%%%%%%%%%%%%%%%%%%%%%%%%%%%%%%%%%%%%%%%
\begin{frame}{L'evoluzione del debito italiano e le sue determinanti /3}
\begin{itemize}
\item Il grafico illustra l'andamento di $n$ (crescita del PIL nominale) e $i$ (tasso di interesse medio sulle emissioni di titoli di Stato)
\end{itemize}

\begin{center}
\centering
\includegraphics[width=\textwidth]{./figure/interesse-crescita-Italy-color.pdf}
\end{center}
\end{frame}

%%%%%%%%%%%%%%%%%%%%%%%%%%%%%%%%%%%%%%%%%%%%
\begin{frame}{Insolvenza e sostenibilità del debito}
\begin{itemize}
\item Si parla di \alert{insolvenza} se lo Stato non è in condizioni di onorare i propri
impegni (pagare gli interessi, rimborsare i titoli a scadenza\ldots{})
\item Si parla di \alert{default} se lo Stato non paga interessi e titoli a scadenza: il
default può essere volontario, a prescindere dalla capacità di pagare.
\item Una \alert{ristrutturazione} del debito è una modifica, solitamente negoziata coi
creditori, delle scadenze e dei pagamenti.
\item La \alert{solvibilità} non è facile da determinare in concreto. Si preferisce
parlare di \alert{sostenibilità} del debito. La sostenibilità corrisponde a
un'elevata probabilità che uno Stato risulti solvibile.
\item La solvibilità è influenzata da:
\begin{itemize}
\item struttura per scadenze dei titoli di Stato;
\item identità dei creditori: se sono risparmiatori, investitori istituzionali,
organizzazioni internazionali, altri Stati\ldots{}
\item valuta di denominazione del debito (es. molti paesi economicamente meno
avanzati si indebitano in dollari o altra valuta): ciò espone il paese al
rischio di una crisi della bilancia dei pagamenti e alle fluttuazioni del
tasso di cambio.
\end{itemize}
\end{itemize}
\end{frame}
%%%%%%%%%%%%%%%%%%%%%%%%%%%%%%%%%%%%%%%%%%%%
\begin{frame}{Fattori che incidono sulla sostenibilità del debito}
\begin{itemize}
\item Se fossimo in grado di identificare il livello massimo di avanzo primario
\(\hat a\), il limite sarebbe
\begin{equation*}
  \hat{b}=\frac{1+n}{i-n}\hat{a}
\end{equation*}
visto che un
livello di \(b\) superiore non potrebbe essere «stabilizzato» e quindi si
determinerebbe una dinamica divergente del rapporto debito/PIL.

\item Tuttavia, non è ovvio cosa possa limitare la capacità di fissare \(a\) al
livello necessario. Non è solo questione di capacità, anche di volontà
politica.

\item L'analisi della sostenibilità deve tenere conto dei possibili shock. Tra
essi:
\begin{itemize}
\item shock macroeconomici
\item cambiamenti delle aspettative degli investitori.
\end{itemize}
\end{itemize}
\end{frame}


%%%%%%%%%%%%%%%%%%%%%%%%%%%%%%%%%%%%%%%%%%%%
\begin{frame}{Quali politiche per la riduzione del debito?}
\begin{itemize}
\item Crescita economica (aumentare $g$)
\item Realizzazione di avanzi primari (aumentare $a$) riducendo le spese o
aumentando le imposte
\item Inflazione non anticipata (aumentare $\pi$, e quindi $n$, senza aumentare
$i$)
\item "Repressione finanziaria" (ridurre $i$) incoraggiando o forzando l'acquisto
di titoli di stato da parte di intermediari finanziari o famiglie
\item Default del debito
\end{itemize}

Ovviamente, non tutte queste strade sono equivalenti. Alcune di esse possono
non essere attuabili o possono comportare costi elevati per il Paese.
\end{frame}

%%%%%%%%%%%%%%%%%%%%%%%%%%%%%%%%%%%%%%%%%%%%
\begin{frame}{Il debito italiano dall'Unità d'Italia al 2017}
\begin{center}
\centering
\includegraphics[width=\textwidth]{./figure/debito-PIL-1861-2017.pdf}
\end{center}
\end{frame}


%%%%%%%%%%%%%%%%%%%%%%%%%%%%%%%%%%%%%%%%%%%%
\begin{frame}{L'inflazione e il debito}
\begin{itemize}
\item Consideriamo l'effetto di un aumento dell'inflazione $\pi$:
\begin{itemize}
\item l'aumento dei prezzi si riflette direttamente sulla
crescita nominale:  $1+n^*=(1+n)(1+\pi)$
\item nella misura in cui è anticipata ($\pi^e$), la maggiore inflazione si riflette sui
tassi di interesse $1+i\,^*=(1+i)(1+\pi^e)$
\end{itemize}
\item Calcoliamo dunque i nuovi valori dei tassi di interesse e di crescita:
\begin{equation*}
  \frac{i\,^*-n^*}{1+n^*} = \frac{1+i\,^*}{1+n^*}-1 = \frac{(1+i)(1+\pi^e)}{(1+n)(1+\pi)}-1
\end{equation*}
\item Quando $\pi^e<\pi$ (l'inflazione non è interamente anticipata):
\begin{equation*}
  \frac{i\,^*-n^*}{1+n^*} < \frac{i-n}{1+n}.
\end{equation*}
\end{itemize}
\end{frame}



\section{I vincoli alla politica di bilancio}

%%%%%%%%%%%%%%%%%%%%%%%%%%%%%%%%%%%%%%%%%%%%
\begin{frame}{Le ragioni dei limiti di bilancio}
\begin{itemize}
\item I politici tendono a operare in un orizzonte di breve periodo e l'elettorato
è poco informato o è, a sua volta, «miope» rispetto agli effetti di lungo
periodo delle scelte politiche.
\item I beneficiari delle spese sono spesso gruppi limitati di individui, mentre
il debito si ripartisce sulla collettività e sulle generazioni future.
\item In un'unione monetaria i singoli paesi possono essere deresponsabilizzati
dalla prospettiva di un intervento della Banca centrale a sostegno del
debito. A questo riguardo, le clausole che vietano il finanziamento diretto
dei debiti e la proibizione di «accesso privilegiato» alle isstituzioni
finanziarie non sembrano deterrenti sufficienti.
\end{itemize}
\begin{block}{}
La crisi del 2008 ha tuttavia mostrato come troppa poca attenzione fosse stata
posta ai \alert{debiti privati}, per salvare i quali molti paesi hanno compromesso
le rispettive posizioni debitorie «sane» (es. Irlanda e Spagna), e agli
\alert{squilibri della bilancia dei pagamenti}
\end{block}
\end{frame}

%%%%%%%%%%%%%%%%%%%%%%%%%%%%%%%%%%%%%%%%%%%%
\begin{frame}{L'equilibrio di bilancio: la Costituzione}
Nel 2012, a seguito degli impegni internazionali presi con il \alert{fiscal compact}, il Parlamento ha modificato l'art. 81 della Costituzione, introducendo il principio dell'\alert{equilibrio di bilancio}. Al Comma 1:

\begin{quoting}
\footnotesize
Lo Stato assicura l’equilibrio tra le entrate e le spese del proprio
bilancio, tenendo conto delle fasi avverse e delle fasi favorevoli del ciclo
economico.

Il ricorso all’indebitamento è consentito solo al fine di considerare gli
effetti del ciclo economico e, previa autorizzazione delle Camere adottata a
maggioranza assoluta dei rispettivi componenti, al verificarsi di eventi
eccezionali.
\end{quoting}

Il Comma 6 dell'Art. 81 rinvia, per i criteri, a una legge approvata con
maggioranza qualificata. La L. 243/2012 per l'attuazione del «pareggio di
bilancio» identifica l'equilibrio con il conseguimento dell'\alert{obiettivo di
medio termine}, ovvero:

\begin{quoting}
\small
«il valore del \alert{saldo strutturale} individuato sulla base dei criteri
stabiliti dall'ordinamento dell'Unione europea».
\end{quoting}

La legge rinvia in modo esplicito al \alert{Patto di stabilità e crescita} della UE.
\end{frame}


%%%%%%%%%%%%%%%%%%%%%%%%%%%%%%%%%%%%%%%%%%%%
\begin{frame}{Il Patto di Stabilità e Crescita}
\begin{itemize}
\item Il quadro di riferimento europeo per le politiche di bilancio è dato dal
\alert{Patto di Stabilità e Crescita} (\emph{Stability and Growth Pact -- SGP}),
sottoscritto nel 1997 e successivamente rafforzato. Esso prevede
\begin{itemize}
\item un \alert{braccio preventivo}, alla base del quale c'è la definizione per
ciascun paese di uno specifico \alert{obiettivo di medio termine} (\emph{medium term
objective - MTO})
\item un \alert{braccio correttivo} reso operativo dalla \alert{Procedura per deficit eccessivo}
(\emph{Excessive Deficit Procedure – EDP}) che si applica in caso di violazione
della regola del deficit o di quella del debito
\end{itemize}
\end{itemize}
\begin{block}{Le basi normative}
\scriptsize\addtolength{\itemsep}{-10pt}
\begin{itemize}
\item gli articoli →\href{http://eur-lex.europa.eu/LexUriServ/LexUriServ.do?uri=CELEX:12008E121:EN:NOT}{121} e →\href{http://eur-lex.europa.eu/LexUriServ/LexUriServ.do?uri=CELEX:12008E126:EN:NOT}{126} del Trattato (TFEU)
\item il →\href{http://europa.eu/legislation\_summaries/economic\_and\_monetary\_affairs/stability\_and\_growth\_pact/l25019\_en.htm}{Regolamento 1466/97} e relative modifiche
\item il \alert{Six Pack}, in vigore dal 13/12/2011, che riforma e rafforza la
legislazione secondaria, introducendo il \alert{semestre europeo}
\item il \alert{Fiscal compact}, contenuto nel trattato intergovernativo (vincolante per
chi lo ha sottoscritto e per chi adotterà l'euro) su stabilità,
coordinamento e governance (TSCG) in vigore dal 1/1/2013
\item il \alert{Two Pack}, in vigore dal 30/5/2013, che introduce ulteriori strumenti di
sorveglianza e monitoraggio
\end{itemize}
→\href{http://ec.europa.eu/economy\_finance/economic\_governance/sgp/legal\_texts/index\_en.htm}{Stability and Growth Pact sul sito della Commissione}
\end{block}
\end{frame}
%%%%%%%%%%%%%%%%%%%%%%%%%%%%%%%%%%%%%%%%%%%%
\begin{frame}{Gli effetti del ciclo sul saldo di bilancio}
\begin{itemize}
\item Il ciclo economico influenza i saldi di finanza pubblica. Infatti:
\begin{itemize}
\item il livello di attività economica influenza le entrate fiscali e, in certa
misura, la spesa pubblica (vedi sussidi di disoccupazione): in situazioni di
crisi l'operare di \alert{stabilizzatori automatici} crea un deficit anche a
legislazione invariata
\item i saldi sono definiti in \% del PIL, a parità di entrate/uscita il rapporto
risente di variazioni del PIL
\end{itemize}
\item Il saldo di bilancio (lo indichiamo come «saldo effettivo» per distinguerlo
dal «saldo strutturale») è:
\begin{equation*}
  \text{saldo effettivo}\quad s(Y)=\frac{T(Y)-G(Y)}{Y}
\end{equation*}
con $T(Y)$ decrescente e $G(Y)$ crescente rispetto al reddito, $s(Y)$ è
funzione crescente di $Y$.
\item Perseguire il pareggio di bilancio in una recessione rappresenta una manovra
\alert{prociclica}, che può accentuare la riduzione del PIL.
\end{itemize}
\end{frame}

%%%%%%%%%%%%%%%%%%%%%%%%%%%%%%%%%%%%%%%%%%%%
\begin{frame}{Saldo effettivo e saldo strutturale}
\begin{itemize}
\item Indichiamo con $s(Y^*)$ il \alert{saldo strutturale}, che si otterrebbe in
corrispondenza del livello del PIL in piena occupazione (o PIL potenziale)
$Y^*$. Chiaramente: $s(Y)>s(Y^*)$ se e solo se $Y>Y^*$.
\item Ipotizzando che $T(Y)=tY$ e $G(Y)=G^*$, abbiamo:
\begin{equation*}
  s(Y)=\frac{tY - G^*}{Y} \quad\implies\quad s'(Y)=\frac{G^*}{Y^2}
\end{equation*}
\item Dunque:
\begin{equation*}
s(Y)\approx s(Y^*)+(Y-Y^*)s'(Y^*)=s(Y^*)+\frac{Y-Y^*}{Y^*}\frac{G^*}{Y^*}
\end{equation*}
dove $\frac{Y-Y^*}{Y^*}$ è detto \emph{output gap} e $G^*/Y^*$ varia da paese a
paese in base alla dimensione della spesa pubblica. In Italia è 0,55.
\begin{equation*}
\text{[saldo strutturale]} = \text{[saldo corrente]} -  0,\!55 \times\text{[output gap]}
\end{equation*}
\item Quando l'economia è sotto il suo potenziale (\emph{output gap} negativo) il saldo
strutturale risulta dunque migliore del saldo effettivo. Il perseguimento del
pareggio strutturale riduce la necessità di manovre procicliche.
\end{itemize}
\end{frame}


%%%%%%%%%%%%%%%%%%%%%%%%%%%%%%%%%%%%%%%%%%%%
\begin{frame}{Il «braccio preventivo» del Patto di stabilità e crescita}
\begin{itemize}
\item \alert{Obiettivo di medio termine}: prevede che il paese sia in pareggio
strutturale o \emph{prossimo al} pareggio strutturale, o sia su un sentiero di
convergenza verso tale obiettivo.
\item Il raggiungimento dell'obiettivo è rafforzato dalla \alert{regola della spesa},
che prevede un limite alla crescita annua della spesa pubblica \alert{primaria},
calcolata al netto di alcune spese legate al ciclo (es. sussidi di
disoccupazione) e della spesa per i programmi europei.
\begin{itemize}
\item La spesa può essere aumentata oltre tale limite solo in presenza di un
aumento corrispondente delle entrate (un aumento «discrezionale», cioè
dovuto a un esplicito cambiamento legislativo, non automatico).
\item Il limite di crescita della spesa è pari alla crescita media del PIL
potenziale, corretta da un fattore che dipende dalla distanza
dall'obiettivo di medio termine.
\end{itemize}
\item Il mancato rispetto di tali obiettivi comporta l'avvio di una \alert{procedura per
deviazione significativa} (\emph{Significant Deviation Procedure}), con l'obbligo
per il Paese di attuare azioni correttive, pena l'applicazione di sanzioni.
\end{itemize}
\end{frame}
%%%%%%%%%%%%%%%%%%%%%%%%%%%%%%%%%%%%%%%%%%%%
\begin{frame}{Il «braccio correttivo» del Patto di stabilità e crescita}
Ciascun paese è tenuto inoltre a rispettare i vincoli su deficit corrente e debito fissati nel Trattato di Maastricht (Art. 126 TFUE):
\begin{itemize}
\item \alert{Limite al deficit corrente} (indebitamento netto): non può eccedere il 3\%
del PIL, a meno che il superamento non sia eccezionale e temporaneo o non sia
diminuito e in avvicinamento a tale valore.
\item \alert{Limite al rapporto debito/PIL} che non deve superare il 60\%. Si intende
rispettato se tale rapporto si sta «riducendo in misura sufficiente» e si sta
avvicinando al valore di riferimento «con ritmo adeguato»:
\begin{itemize}
\item \alert{Regola del debito}: la riduzione annua del rapporto
debito/PIL deve essere almeno pari a 1/20 della differenza tra debito
corrente e limite del 60\%.
\end{itemize}
\item Il mancato rispetto di tali limiti comporta l'apertura di una \alert{procedura per
disavanzo eccessivo} (\emph{Excessive Deficit Procedure}, EDP). Anche in questo
caso dovranno essere attuate azioni correttive, pena l'applicazione di
sanzioni.
\end{itemize}
\begin{block}{}
\small
La decisione sull'apertura delle procedure di violazione non è automatica,
spetta al Consiglio dei ministri dell'economia e finanze dei paesi UE
(ECOFIN), su proposta della Commissione.
\end{block}
\end{frame}

%%%%%%%%%%%%%%%%%%%%%%%%%%%%%%%%%%%%%%%%%%%%
\begin{frame}{La flessibilità dei vincoli europei}
\begin{itemize}
\item Le regole descritte non sono applicate in modo meccanico è rigido. È
prevista una certa flessibilità per tenere conto di circostanze specifiche e
consentire spazi fiscali per realizzare riforme:
\begin{itemize}
\item in presenza di recessione, a seconda della gravità della stessa, gli
aggiustamenti di bilancio richiesti sono di entità inferiore;
\item in presenza di «riforme strutturali» e programmi di investimenti approvati
dalla UE è concesso un maggiore spazio di bilancio.
\end{itemize}
\item In presenza di situazioni eccezionali si possono sospendere temporaneamente
le regole europee (\alert{clausola di salvaguardia generale})
\begin{itemize}
\item Tale clausola è stata applicata nella primavera del 2020. Per consentire
di intraprendere le azioni energiche richieste dalla crisi pandemica, le
regole del braccio correttivo e del braccio preventivo sono rimaste
sospese fino a tutto il 2023.
\end{itemize}
\end{itemize}
\end{frame}

%%%%%%%%%%%%%%%%%%%%%%%%%%%%%%%%%%%%%%%%%%%%
\begin{frame}{La riforma del Patto di stabilità}
\begin{itemize}
\item Insoddisfazione per l'attuale sistema di regole:
\begin{itemize}
\item regole troppo complesse;
\item troppa discrezionalità della Commissione e del Consiglio che conduce a una
«politicizzazione» delle decisioni;
\item regole uniformi per paesi con situazioni diverse.
\end{itemize}
\item Nello specifico:
\begin{itemize}
\item la stima dell'\emph{output gap} è soggetta ad ampi margini di errore e la
stessa nozione di \emph{output gap} viene contestata dal punto di vista
teorico. Al di là delle intenzioni, non si elimina l'effetto prociclico
delle politiche di bilancio.
\item le regole correnti, non distinguendo tra spese correnti e investimenti,
finoscono per scoraggiare questi ultimi, più facilmente
rinviabili. Sarebbe desiderabile una \alert{golden rule}, ovvero una regola che
limitasse le sole spese correnti, escludendo dal vincolo le spese in conto
capitale.
\end{itemize}
\end{itemize}
\end{frame}


%%%%%%%%%%%%%%%%%%%%%%%%%%%%%%%%%%%%%%%%%%%%
\begin{frame}{La riforma del Patto di stabilità /2}
\begin{itemize}
\item Il progetto di riforma attualmente in discussione:
\begin{itemize}
\item semplificazione, con adozione di un'unica regola, che ricalca la \alert{regola
della spesa} ed è fissata in funzione dell'obiettivo di riduzione del
debito pubblico.
\item traiettorie di riduzione del debito definite Paese per Paese in base alla
specifica situazione economica e fiscale;
\item più spazio agli investimenti;
\item più automatismo nel definire una violazione delle regole;
\end{itemize}
\end{itemize}

\begin{figure}
\centering
\includegraphics[width=10cm]{./figure/titolo-Sole-su-patto-stabilità.png}
\end{figure}
\scriptsize
\emph{Il Sole 24 Ore} del 9/12/23
\end{frame}

%%%%%%%%%%%%%%%%%%%%%%%%%%%%%%%%%%%%%%%%%%%%
\begin{frame}{Il «semestre europeo»}
\begin{itemize}
\item Introdotto nel 2010 per favorire il coordinamento a livello europeo
preliminarmente alla programmazione di bilancio a livello nazionale:
\begin{itemize}
\item il quadro previsionale macroeconomico adottato dai diversi paesei è definito
in modo omogeneo a livello europeo;
\item si ha una valutazione preventiva degli obiettivi definiti dai paesi.
\end{itemize}
\end{itemize}

\begin{figure}[htbp]
\centering
\includegraphics[height=5cm]{./figure/semestre-europeo.png}
\end{figure}
\end{frame}


\section{Il ciclo di bilancio e la manovra di finanza pubblica}

%%%%%%%%%%%%%%%%%%%%%%%%%%%%%%%%%%%%%%%%%%%%
\begin{frame}{Il ciclo dei documenti di finanza pubblica (ex L. 196/2009, art. 7)}
\begin{itemize}
\item Il \alert{Documento di Economia e Finanza (DEF)} (→\href{http://www.mef.gov.it/documenti-pubblicazioni/doc-finanza-pubblica/index.html}{vedi sul sito del MEF}) illustra
la situazione economico-finanziaria del Paese e gli obiettivi che il Governo
intende raggiungere. Comprende:
\begin{itemize}
\item \alert{Sez.I - Programma di Stabilità}. Indica il quadro delle previsioni
economico-finanziarie e gli obiettivi relativi ai principali saldi di
finanza pubblica per il triennio successivo
\item \alert{Sez.II - Analisi e tendenze della finanza pubblica.} I conti pubblici
dell'anno precedente e le previsioni, dettagliati per sottosettori di
spesa
\item \alert{Sez.III - Programma nazionale di riforma.} Indica lo stato di
avanzamento delle riforme avviate, gli squilibri macroeconomici e i
fattori che incidono sulla competitività, le riforme da attuare e il
loro prevedibile impatto
\end{itemize}
È presentato alle Camere entro il 10 aprile, così da
consentire l’approvazione e l’invio, entro il 30 aprile, delle sezioni
relative al Programma di Stabilità (PS) e al Piano Nazionale di Riforma
(PNR) al Consiglio dell’Ue.
\end{itemize}
\end{frame}

%%%%%%%%%%%%%%%%%%%%%%%%%%%%%%%%%%%%%%%%%%%%
\begin{frame}{Il ciclo dei documenti di finanza pubblica (continua)}
\begin{itemize}
\item La \alert{Nota di aggiornamento al DEF} contiene gli eventuali
  aggiornamenti degli obiettivi programmatici fissati nel DEF anche al fine di
  recepire le raccomandazioni formulate dal Consiglio dell’Unione Europea. È
  presentata dal Governo al Parlamento entro il 20 settembre.
\item Il \alert{Documento Programmatico di Bilancio (DPB)} riprende gli
  obiettivi programmatici contenuti nella Nota di aggiornamento al DEF ed
  illustra le misure inserite nella manovra di bilancio. È trasmesso entro il
  15 ottobre
  alla Commissione Europea e all'Eurogruppo \\[0pt]
  \emph{In attuazione del Regolamento UE n.473/2013 (Two Pack)}
\item Il \alert{DDL di Bilancio}. La sua presentazione entro il 15 ottobre dà
  inizio alla sessione parlamentare di bilancio. Deve essere approvato dal
  Parlamento entro il 31 dicembre.
\item Il \alert{Rendiconto generale} rileva e riassume i risultati ottenuti
  nel corso dell’anno precedente. È presentato dal Governo al Parlamento per
  approvazione, previa verifica della Corte dei conti, entro il 30 giugno.
\end{itemize}
\end{frame}

%%%%%%%%%%%%%%%%%%%%%%%%%%%%%%%%%%%%%%%%%%%%
\begin{frame}{I saldi di bilancio corretti per il ciclo}
\vspace{-5mm}
\begin{figure}
\centering
\includegraphics[width=\textwidth]{./figure/NADEF-2023-Tabella-III-5.png}
\end{figure}

\vspace{-3mm}
\tiny Fonte: →\href{https://www.dt.mef.gov.it/export/sites/sitodt/modules/documenti\_it/analisi\_progammazione/documenti\_programmatici/nadef\_2023/NADEF-2023.pdf}{Nota di aggiornamento al DEF 2023}
\end{frame}

%%%%%%%%%%%%%%%%%%%%%%%%%%%%%%%%%%%%%%%%%%%%
\begin{frame}{Gli scostamenti rispetto alle regole UE}
\vspace{-3.5mm}
\begin{figure}[htbp]
\centering
\includegraphics[height=7.5cm]{./figure/NADEF-2023-Tabella-III-6.png}
\end{figure}

\vspace{-3mm}
\tiny Fonte: →\href{https://www.dt.mef.gov.it/export/sites/sitodt/modules/documenti\_it/analisi\_progammazione/documenti\_programmatici/nadef\_2023/NADEF-2023.pdf}{Nota di aggiornamento al DEF 2023}
\end{frame}


%%%%%%%%%%%%%%%%%%%%%%%%%%%%%%%%%%%%%%%%%%%%
\begin{frame}{La manovra per il 2024}
\begin{columns}
\begin{column}{.75\columnwidth}
\vspace{-3mm}
\begin{center}
\includegraphics[height=7.5cm]{./figure/audizione-UPB-DDL-bilancio-2024-tab-3-1.png}
\end{center}
\end{column}

\begin{column}{.25\columnwidth}
\footnotesize
N.B. L'entità della manovra di bilancio è indicata come
differenza rispetto alle previsioni tendenziali \alert{a legislazione vigente}

\vspace{5mm}

\tiny Fonte: →\href{https://www.upbilancio.it/wp-content/uploads/2021/11/Audizione-UPB-DDL-bilancio-2022.pdf}{Audizione del 23/11/2021 del Presidente dell'Ufficio Parlamentare di Bilancio} (Tab. 2.1)
\end{column}
\end{columns}
\end{frame}
\end{document}