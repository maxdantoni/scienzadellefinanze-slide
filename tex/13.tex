% Intended LaTeX compiler: xelatex
\documentclass[11pt,italian]{beamer}
\usepackage{graphicx}
\usepackage{longtable}
\usepackage{wrapfig}
\usepackage{rotating}
\usepackage[normalem]{ulem}
\usepackage{amsmath}
\usepackage{amssymb}
\usepackage{capt-of}
\usepackage{hyperref}
\institute{Università di Siena}
\usepackage{localheader}
\usepackage{tikz}
\usepackage{booktabs,tabularx,tabularray}
\usepackage{setspace}
\usepackage{quoting}
\usepackage[italian]{babel}
\usepackage{fancybox}
\usepackage{tabularray}
\newcolumntype{R}{>{\raggedleft\arraybackslash}X}
\usetheme{default}
\author{Massimo D'Antoni}
\date{2023-2024}
\title{La tassazione dei beni e servizi}
\subtitle{Scienza delle Finanze}
\hypersetup{
 pdfauthor={Massimo D'Antoni},
 pdftitle={La tassazione dei beni e servizi},
 pdflang={Italian}}
\begin{document}

\maketitle


\section{Imposte sui beni e servizi}

%%%%%%%%%%%%%%%%%%%%%%%%%%%%%%%%%%%%%%%%%%%%
\begin{frame}{Classificazione}
\begin{center}
\includegraphics[width=\textwidth]{./figure/imposte-beni-servizi.pdf}
\end{center}
\end{frame}

%%%%%%%%%%%%%%%%%%%%%%%%%%%%%%%%%%%%%%%%%%%%
\begin{frame}{Evoluzione del gettito delle imposte sui beni e servizi}
\begin{center}
\includegraphics[width=\textwidth]{./figure/gettito-imposte-beni-OCSE-color.pdf}
\end{center}
\end{frame}

%%%%%%%%%%%%%%%%%%%%%%%%%%%%%%%%%%%%%%%%%%%%
\begin{frame}{Le imposte generali sul consumo: le varie fasi del processo produttivo}
\begin{itemize}
\item Le imposte possono gravare sulle diverse fasi del processo di produzione e
\end{itemize}
vendita. Ipotizziamo 3 fasi: produzione, ingrosso, dettaglio.

\begin{center}
\includegraphics[width=\textwidth]{./figure/schema-imposte-generali.pdf}
\end{center}
\end{frame}

%%%%%%%%%%%%%%%%%%%%%%%%%%%%%%%%%%%%%%%%%%%%
\begin{frame}{Le imposte plurifase sul valore pieno}
\begin{itemize}
\item Nel caso di un'imposta sul valore pieno, i prezzi ai quali sono commisurate
le imposte (determinate su base netta) in ciascuna fase sono:
$$ p_A=A \qquad p_B=B+q_A \qquad p_C=C+q_C $$
\item Visto che $q_i=(1+\tau_i)p_i$, procedendo per sostituzioni successive:
\begin{equation*}
\begin{split}
  \tau_Ap_A&= \tau_AA \\
  \tau_Bp_B&= \tau_B(B + q_A) = \tau_B\big[B+ (1+t_A)A\big] \\
  \tau_Cp_C&= \tau_C(C + q_B) = \tau_C\Big[C+(1+\tau_B)\big[B+ (1+t_A)A\big]\Big].
\end{split}
\end{equation*}
da cui calcoliamo l'imposta totale:
\begin{equation*}
T^{PC} = \tau_CC + \Big[\tau_B+\tau_C(1+\tau_B)\Big]B
+ \Big[\tau_A + \tau_B(1+\tau_A) + \tau_C(1+\tau_B)(1+\tau_A)\Big]A 
\end{equation*}
\item Con $\tau$ uniforme si evidenzia il carattere «cumulativo» dell'imposta:
\begin{equation*}
T^{PC}=\tau\, C + \tau\,\big[2+\tau\big]\, B + \tau\,\big[3 + \tau(3+\tau)\big]\, A.
\end{equation*}
\end{itemize}
\end{frame}

%%%%%%%%%%%%%%%%%%%%%%%%%%%%%%%%%%%%%%%%%%%%
\begin{frame}{Le imposte plurifase sul valore pieno: l'incentivo all'integrazione verticale}
\begin{itemize}
\item Nell'ipotesi $A=B=C=100$ e $\tau=10\%$ (uniforme):
\begin{center}
\includegraphics[width=.8\textwidth]{./figure/imposte-plurifase-esempio-1.png}
\end{center}
L'aliquota effettiva su base netta è: $64,1/300=21,36\%$

\item Se c'è integrazione verticale di Produzione e Ingrosso:
\begin{center}
\includegraphics[width=.8\textwidth]{./figure/imposte-plurifase-esempio-2.png}
\end{center}
L'aliquota effettiva su base netta è: $52/300=17,3\%$
\end{itemize}
\end{frame}

%%%%%%%%%%%%%%%%%%%%%%%%%%%%%%%%%%%%%%%%%%%%
\begin{frame}{L'imposta monofase}
\begin{itemize}
\item Per evitare gli inconvenienti di un'imposta cumulativa possiamo applicare
l'imposta sul valore pieno una sola volta, in uno dei tre stadi:
\begin{itemize}
\item cessione al grossista: $\tau A$
\item cessione al dettagliante: $\tau(A+B)$
\item cessione al consumatore finale: $\tau(A+B+C)$
\end{itemize}
\item Nel caso di cessione al consumatore finale l'aliquota effettiva è $\tau$,
negli altri casi è inferiore. Ad esempio, nel caso in cui ad essere tassata
sia la cessione al dettagliante, l'aliquota effettiva è:
$$ \frac{\tau(A+B)}{A+B+C}<\tau $$
\item Un esempio di monofase: le \emph{sales taxes} americane, applicate dai singoli
stati nella fase di vendita al dettaglio.
\end{itemize}
\end{frame}

%%%%%%%%%%%%%%%%%%%%%%%%%%%%%%%%%%%%%%%%%%%%
\begin{frame}{L'imposta sul valore aggiunto}
\begin{itemize}
\item Colpisce in ciascuna fase l'\alert{incremento} di valore del bene scambiato.
\end{itemize}

\begin{figure}[htbp]
\centering
\includegraphics[width=10cm]{./figure/flussi-VA-metaflow-1.pdf}
\end{figure}

\begin{itemize}
\item La somma del valore aggiunto delle imprese è pari al consumo finale, a sua
volta pari al fatturato totale meno gli acquisti di beni intermedi (le
transazioni tra imprese).
\end{itemize}
\end{frame}

%%%%%%%%%%%%%%%%%%%%%%%%%%%%%%%%%%%%%%%%%%%%
\begin{frame}{L'imposta sul valore aggiunto: metodi di calcolo}
Due metodi per la determinazione della base imponibile:
\begin{itemize}
\item \alert{Base da base (\emph{subtraction method})}.
\begin{itemize}
\item L'imposta non si applica alle singole transazioni, ma al valore aggiunto
complessivo (differenza tra totale vendite e totale acquisti) dei soggetti
che effettuano vendite di beni e servizi.
\item L'imposta appare formalmente simile a un'imposta diretta, applicata alla
remunerazione complessiva dei fattori.
\item L'aliquota è definita \alert{su base lorda}.
\end{itemize}

\item \alert{Imposta da imposta (\emph{invoice credit method})}.
\begin{itemize}
\item L'imposta è calcolata su ogni singola transazione. In fattura sono
indicati prezzo netto, imposta e prezzo lordo.
\item Formalmente è a carico dell'acquirente, ma viene versata dal venditore che
l'ha incassata.
\item L'aliquota è definita \alert{su base netta}.
\end{itemize}
\end{itemize}
\end{frame}

%%%%%%%%%%%%%%%%%%%%%%%%%%%%%%%%%%%%%%%%%%%%
\begin{frame}{L'imposta sul valore aggiunto col metodo base da base}
\begin{itemize}
\item Calcoliamo l'imposta (indicando con $t$ le aliquote su base lorda):
\begin{equation*}
T^{BB} = t_Aq_A + t_B(q_B-q_A) + t_C(q_C-q_B).
\end{equation*}
\item Visto che:
\begin{equation*}
A=(1-t_A)q_A\qquad B=(1-t_B)(q_B-q_A)\qquad C=(1-t_C)(q_C-q_B)
\end{equation*}
possiamo scrivere:
\begin{equation*}
  T^{BB} = \frac{t_A}{1-t_A}A + \frac{t_B}{1-t_B}B +\frac{t_C}{1-t_C}C.
\end{equation*}
formula che evidenzia l'assenza di effetti «cumulativi».
\item Inoltre, se $t$ è uniforme le formule si semplificano:
\begin{equation*}
  T^{BB} = t\,q_C = \frac{t}{1-t}(A + B+ C).
\end{equation*}
per cui l'aliquota effettiva è $t/(1-t)$, aliquota «su base netta»
corrispondente all'aliquota legale «su base lorda» $t$.
\end{itemize}
\end{frame}

%%%%%%%%%%%%%%%%%%%%%%%%%%%%%%%%%%%%%%%%%%%%
\begin{frame}{L'imposta sul valore aggiunto col metodo base da base: esempio}
\begin{itemize}
\item Con aliquota uniforme $t=20\%$:
\begin{center}
\includegraphics[width=.8\textwidth]{./figure/esempio-base-da-base-aliquota-uniforme.png}
\end{center}
\begin{itemize}
\item L'imposta complessiva è 75€, dunque l'aliquota effettiva su base lorda è
$75/375=20\%$, quella su base netta è $75/300=25\%$.
\end{itemize}

\item Se l'aliquota non è uniforme (es. 20\% nelle fasi di produzione e ingrosso,
10\% al dettaglio):
\begin{center}
\includegraphics[width=.8\textwidth]{./figure/esempio-base-da-base-aliquota-differenziata.png}
\end{center}
\begin{itemize}
\item L'imposta complessiva è 61,1€, l'aliquota effettiva su base \emph{lorda} è
$61,1/361,1=16,92\%$, maggiore dell'aliquota pagata dal consumatore
finale. L'aliquota effetti va su base netta è $61,1/361,1=20,36\%$.
\end{itemize}
\end{itemize}
\end{frame}


%%%%%%%%%%%%%%%%%%%%%%%%%%%%%%%%%%%%%%%%%%%%
\begin{frame}{L'imposta sul valore aggiunto col metodo imposta da imposta}
\begin{itemize}
\item Indichiamo con $\tau_A$, $\tau_B$ e $\tau_C$ le aliquote, su base netta,
applicate nelle tre fasi al prezzo al netto dell'imposta.
\item Le imposte versate sono:
\begin{equation*}
\tau_Ap_A \qquad \tau_Bp_B-\tau_Ap_A \qquad \tau_Cp_C-\tau_Bp_B
\end{equation*}
per cui:
\begin{equation*}
  T^{II}=\tau_Ap_A + (\tau_Bp_B-\tau_Ap_A) +
  (\tau_C p_C-\tau_Bp_B)=\tau_Cp_C.
\end{equation*}
\item Il valore aggiunto al costo dei fattori è:
\begin{equation*}
A=p_A \qquad B = p_B-p_A \qquad C=p_C-p_B
\end{equation*}
dunque:
\begin{equation*}
\begin{split}
  T^{II}&=\tau_AA + [\tau_B(A+B)-\tau_AA] + [\tau_C(A+B+C)-\tau_B(A+B)]\\
  &=\tau_C(A+B+C).
\end{split}
\end{equation*}
\item L'imposta effettiva coincide con l'aliquota applicata al consumatore finale.
\end{itemize}
\end{frame}

%%%%%%%%%%%%%%%%%%%%%%%%%%%%%%%%%%%%%%%%%%%%
\begin{frame}{L'imposta sul valore aggiunto col metodo imposta da imposta: esempio}
\begin{itemize}
\item Con aliquota uniforme $\tau=25\%$ (corrispondente a un'aliquota su base
lorda del 20\%):
\begin{center}
\includegraphics[width=\textwidth]{./figure/esempio-imposta-da-imposta-1.png}
\end{center}
\begin{itemize}
\item L'imposta complessiva è 75€, l'aliquota effettiva è: $75/300=25\%$,
esattamente come nel precedente esempio di applicazione del metodo base da
base
\end{itemize}
\end{itemize}
\end{frame}


%%%%%%%%%%%%%%%%%%%%%%%%%%%%%%%%%%%%%%%%%%%%
\begin{frame}{L'imposta sul valore aggiunto col metodo imposta da imposta: esempio}
\begin{itemize}
\item Con aliquota differenziata: \$$\tau$\textsubscript{A}=$\tau$\textsubscript{B}$=25\%$ e $\tau_C=11,1\%$
(quest'ultima corrisponde a un'aliquota su base lorda del 10\%)
\begin{center}
\includegraphics[width=\textwidth]{./figure/esempio-imposta-da-imposta-2.png}
\end{center}
\begin{itemize}
\item Il dettagliante ha diritto al rimborso dell'IVA pagata al grossista in
eccesso sull'IVA a debito.
\item L'imposta complessiva è 33,3€, l'aliquota effettiva è: $33,3/300=11,1\%$.
\end{itemize}
\end{itemize}
\end{frame}

%%%%%%%%%%%%%%%%%%%%%%%%%%%%%%%%%%%%%%%%%%%%
\begin{frame}{Vantaggi e svantaggi del metodo imposta da imposta}
\begin{itemize}
\item Il metodo imposta da imposta presenta diversi vantaggi: trasparenza e
neutralità rispetto all'integrazione verticale anche in presenza di aliquote
differenziate tra le varie fasi.
\item Esso pone tuttavia anche qualche problema in alcuni casi:
\begin{enumerate}
\item \alert{beni rimessi in commercio dopo l'uso}: se un privato vende l'auto
all'officina, che la rivende applicando l'IVA\ldots{}.
\item \alert{servizi resi dal settore finanziario e assicurativo}: difficile
determinare il valore del servizio, remunerato con il margine tra
interessi attivi e passivi;
\item \alert{servizi della Pubblica amministrazione}, forniti gratuitamente o ad un
prezzo inferiore al costo.
\end{enumerate}
\item Nel caso 1 la base imponibile è la differenza tra prezzo di vendita e prezzo
di acquisto dell'auto dal privato («metodo del margine»)
\item Nei casi 2 e 3 la soluzione è l'\alert{esenzione}. Attenzione: c'è differenza tra
operazioni esenti e operazioni non imponibili.
\end{itemize}
\end{frame}

%%%%%%%%%%%%%%%%%%%%%%%%%%%%%%%%%%%%%%%%%%%%
\begin{frame}{Operazioni imponibili IVA: esempio}
\begin{itemize}
\item Su un'operazione non imponibile l'effetto è quello che ci sarebbe con
aliquota IVA pari a zero. Questo perché al venditore, che non applica l'IVA
all'acquirente, viene riconosciuto il credito per l'IVA sugli acquisti.
\item Se è non imponibile una vendita a un consumatore finale (B2C, \emph{business-to-consumer}):
\end{itemize}

\begin{center}
\includegraphics[width=\textwidth]{./figure/esempio-non-imponibile-IVA-B2C.png}
\end{center}
\end{frame}


%%%%%%%%%%%%%%%%%%%%%%%%%%%%%%%%%%%%%%%%%%%%
\begin{frame}{Operazioni imponibili IVA: esempio /2}
\begin{itemize}
\item Se è non imponibile una vendita a un altro soggetto IVA (B2B,
\emph{business-to-business}), ad es. dal grossista al dettagliante:
\end{itemize}

\begin{center}
\includegraphics[width=\textwidth]{./figure/esempio-non-imponibile-IVA-B2B.png}
\end{center}

\begin{itemize}
\item In questo caso la non imponibilità dell'operazione non ha alcun effetto sul
prezzo e sull'imposta finale pagati. L'aliquota effettiva coincide con
l'aliquota legale per il consumatore.
\end{itemize}
\end{frame}

%%%%%%%%%%%%%%%%%%%%%%%%%%%%%%%%%%%%%%%%%%%%
\begin{frame}{Operazioni esenti IVA: esempio}
\begin{itemize}
\item In un'operazione esente chi vende non ha diritto all'IVA a credito sugli
acquisti: il prezzo del bene resta dunque gravato dell'IVA pagata «a monte»
\item Se operazione B2C (\emph{business-to-consumer}) l'imposta esclude il valore
aggiunto $C$: $T^{II}=\tau_A A + [\tau_B(A+B)-\tau_A A] = \tau_B (A+B)$.
\end{itemize}

\begin{center}
\includegraphics[width=\textwidth]{./figure/esempio-esente-IVA-B2C.png}
\end{center}
\end{frame}



%%%%%%%%%%%%%%%%%%%%%%%%%%%%%%%%%%%%%%%%%%%%
\begin{frame}{Operazioni esenti IVA: esempio /2}
\begin{itemize}
\item Se operazione B2B (\emph{business-to-business}), ad es. cessione al dettagliante, abbiamo:
$T^{II}=\tau_A A + \tau_C(A+B+C)$, con l'effetto di tassare due volte $A$.
\end{itemize}

\begin{center}
\includegraphics[width=\textwidth]{./figure/esempio-esente-IVA-B2B.png}
\end{center}
\end{frame}


\section{L'IVA in Italia e in Europa}



%%%%%%%%%%%%%%%%%%%%%%%%%%%%%%%%%%%%%%%%%%%%
\begin{frame}{L'IVA in Italia}
\begin{itemize}
\item Introdotta nel 1973. In precedenza era presenta l'IGE (Imposta Generale
sulle Entrate), un'imposta plurifase sul valore pieno.
\item Il suo \emph{presupposto} è una delle seguenti circostanze:
\begin{itemize}
\item la cessione di beni nel territorio dello stato
\item la prestazione di servizi da parte di soggetti residenti se effettuati
nell'esercizio di imprese o di arti e professioni;
\item gli acquisti intracomunitari e le importazioni, da chiunque effettuati.
\item l'autoconsumo (nell'esercizio di impresa o arti e professioni)
\end{itemize}
Nota bene: il riferimento alla cessione fa sì che l'imposta sia «su base
finanziaria» e non «su base reale» (non sono tassate scorte e rimanenze).

\item \emph{Soggetti passivi} sono gli imprenditori, gli esercenti arti o professioni,
i soggetti che effettuano importazioni o acquisti intracomunitari. I
soggetti IVA sono tenuti a versare l'imposta con \alert{obbligo di rivalsa} sugli
acquirenti.

\item La \emph{base imponibile} è l'ammontare complessivo del corrispettivo per
l'acquisto del bene o del servizio, cui l'aliquota si applica \alert{su base netta}.
\end{itemize}
\end{frame}

%%%%%%%%%%%%%%%%%%%%%%%%%%%%%%%%%%%%%%%%%%%%
\begin{frame}{Aliquote Iva in Italia}
\begin{itemize}
\item Le \emph{aliquote}, la cui determinazione è soggetta a vincoli comunitari,
sono:
\begin{itemize}
\item aliquota \alert{normale} 22\% (era al 20\% fino al 2011)
\item aliquota \alert{ridotta} 10\% (es. carne, pesce, zucchero\dots{}, hotel e
ristoranti, ristrutturazioni edilizie, acqua, farmaceutici, trasporti,
cinema e teatri, gas naturale ed elettricità)
\item aliquota \alert{ridotta} 5\% per prestazioni rese da cooperative sociali
\item aliquota \alert{super-ridotta} 4\% (es. pasta, pane, burro, dispositivi medici
per disabili, libri e periodici, alcune ristrutturazioni---ma anche
licenze TV)
\end{itemize}
\item Sono \alert{esenti} (aliquota zero):
\begin{itemize}
\item servizi di credito e assicurazioni;
\item servizi sanitari, servizi educativi, servizi pubblici di trasporto, alcuni
servizi culturali;
\item i servizi resi da imprese e lavoratori autonomi assoggettati al regime
sostitutivo (ricavi non superiori a 65 mila euro).
\end{itemize}

\item La presenza di aliquote ridotte su beni essenziali determina un (limitato)
effetto di progressività.
\end{itemize}
\end{frame}

%%%%%%%%%%%%%%%%%%%%%%%%%%%%%%%%%%%%%%%%%%%%
\begin{frame}{Aliquote Iva in Europa (2023)}
\begin{columns}
\begin{column}{.5\columnwidth}
\scriptsize
\begin{tabular}{p{1.5cm}ccc}
  \toprule
& ordinaria & ridotta & \parbox{1cm}{\centering super ridotta} \\ 
\midrule
Austria & 20 & 10 / 13 &  \\ 
Belgio & 21 & 6 / 12 &  \\ 
Bulgaria & 20 & 9 &  \\ 
Cipro & 19 & 5 / 9 &  \\ 
Croazia & 25 & 5 / 13 &  \\ 
Danimarca & 25 &  &  \\ 
Estonia & 20 & 9 &  \\ 
Finlandia & 24 & 10 /14 &  \\ 
Francia & 20 & 5,5 / 10 & 2,1 \\ 
Germania & 19 & 7 &  \\ 
Grecia & 24 & 6 / 13 &  \\ 
Irlanda & 23 & 9 / 13,5 & 4,8 \\ 
Italia & 22 & 5 / 10 & 4 \\ 
Lettonia & 21 & 5 / 12 &  \\ 
\bottomrule
\end{tabular}
\end{column}

\begin{column}{.5\columnwidth}
\scriptsize
\begin{tabular}{p{1.5cm}ccc}
  \toprule
& ordinaria & ridotta & \parbox{1cm}{\centering super ridotta} \\ 
\midrule
Lituania & 21 & 5 / 9 &  \\ 
Lussemburgo & 17 & 8 & 3 \\ 
Malta & 18 & 5 / 7 &  \\ 
Paesi Bassi & 21 & 9 &  \\ 
Polonia & 23 & 5 / 8 &  \\ 
Portogallo & 23 & 6 / 13 &  \\ 
Regno Unito & 20 & 5 &  \\ 
Rep. Ceca & 21 & 10 / 15 &  \\ 
Romania & 19 & 5 / 9 &  \\ 
Slovacchia & 20 & 10 &  \\ 
Slovenia & 22 & 5 / 9.5 &  \\ 
Spagna & 21 & 10 & 4 \\ 
Svezia & 25 & 6 / 12 &  \\ 
Ungheria & 27 & 5 / 18 &  \\ 
\bottomrule
\end{tabular}
\end{column}
\end{columns}

\footnotesize
\begin{itemize}
\item Entro il quadro definito dalla normativa europea i paesi UE hanno la
possibilità di fissare le proprie aliquote
\item Le aliquote ridotte sono applicabili solo a categorie esplicitamente
previste da normativa europea (Annex III della Direttiva sull'Iva
2006/112/EC).
\end{itemize}

→ \href{https://ec.europa.eu/taxation\_customs/sites/taxation/files/resources/documents/taxation/vat/how\_vat\_works/rates/vat\_rates\_en.pdf}{VAT Rates Applied in the Member States of the European Union} 
\end{frame}



%%%%%%%%%%%%%%%%%%%%%%%%%%%%%%%%%%%%%%%%%%%%
\begin{frame}{Aliquote Iva in Italia: evoluzione}
\begin{center}
\centering
\includegraphics[height=7cm]{./figure/IVA-aliquote-Italia-color.pdf}
\end{center}
\end{frame}


%%%%%%%%%%%%%%%%%%%%%%%%%%%%%%%%%%%%%%%%%%%%
\begin{frame}{Quali sono i vantaggi dell'IVA?}
\begin{itemize}
\item A differenza di altre imposte generali sugli scambi, l'IVA è \alert{neutrale} e
\alert{trasparente}:
\begin{itemize}
\item non rende conveniente l'integrazione verticale;
\item non influisce sulle scelte degli input.
\end{itemize}
Diamond e Mirrlees (1971): se i mercati sono concorrenziali non è mai
efficiente distorcere i prezzi alla produzione, quale che sia l'obiettivo
redistributivo che si vuole perseguire.
\item Difficoltà di evasione: la determinazione dell'imposta è il risultato di
molteplici operazioni di compravendita, in cui c'è contrasto di interessi
tra le parti. Tuttavia:
\begin{itemize}
\item interesse del consumatore finale a evadere;
\item l'evasione in una fase può sollecitare l'evasione nella fase precedente;
\item il rimborso dell'IVA può essere occasione per frodi fiscali (ad es. la
«frode carosello»).
\end{itemize}
\end{itemize}
\end{frame}

%%%%%%%%%%%%%%%%%%%%%%%%%%%%%%%%%%%%%%%%%%%%
\begin{frame}{L'evasione dell'IVA}
\begin{itemize}
\item L'evasione dell'IVA non è inferiore a quella delle altre imposte
\end{itemize}

\begin{center}
\centering
\includegraphics[height=5cm]{./figure/IVA-tax-gap.pdf}
\end{center}

\begin{itemize}
\item Alcune soluzioni introdotte negli ultimi anni:
\begin{itemize}
\item il \emph{reverse charge};
\item lo \emph{split payment} («scissione dei pagamenti»).
\end{itemize}
\end{itemize}
\end{frame}

%%%%%%%%%%%%%%%%%%%%%%%%%%%%%%%%%%%%%%%%%%%%
\begin{frame}{\emph{Reverse charge}}
\begin{itemize}
\item Con il \emph{reverse charge} (inversione contabile), il versamento non viene
effettuato dal venditore ma dall'acquirente, che contabilizza l'Iva sugli
acquisti sia a debito che a credito (analogamente a quanto accade per le
operazioni intracomunitarie)
\item Lo scopo è scoraggiare l'evasione dell'Iva realizzata nelle fasi intermedie
es. attraverso false fatturazioni per aumentare l'Iva a credito
\item Giudicata efficace, ma
\begin{itemize}
\item modifica in modo rilevante il funzionamento dell'Iva (richiede accordo con
UE)
\item può incentivare evasione nella fase finale, e dunque va accompagnata dal
potenziamento dell'attività di contrasto in tale fase
\end{itemize}
\end{itemize}
\end{frame}


\section{Le imposte speciali sui beni e servizi}


%%%%%%%%%%%%%%%%%%%%%%%%%%%%%%%%%%%%%%%%%%%%
\begin{frame}{Imposte indirette: classificazione amministrativa e gettito (2018)}
\begin{figure}[htbp]
\centering
\includegraphics[height=6cm]{./figure/imposte-indirette-2018.png}
\end{figure}

\begin{itemize}
\item Si tratta di un insieme molto eterogeneo di imposte.
\item Alcune di esse sono anacronistiche, hanno elevati costi amministrativi ed un
gettito limitato.
\end{itemize}
\end{frame}

%%%%%%%%%%%%%%%%%%%%%%%%%%%%%%%%%%%%%%%%%%%%
\begin{frame}{Imposte speciali}
\begin{itemize}
\item Il ruolo delle imposte \alert{speciali} (= applicate a particolari categorie di
beni di consumo) è andato diminuendo nel tempo.
\item Tuttavia, restano alcune \alert{accise} (= imposte speciali e nella maggior parte
dei casi specifiche) che mantengono una certa importanza, in quanto
colpiscono beni il cui consumo lo Stato desidera scoraggiare («sin taxes»):
\begin{itemize}
\item accisa sul tabacco;
\item accisa sulle bevande alcoliche;
\item accisa sui combustibili fossili;
\item accisa sull'energia elettica.
\end{itemize}
In alcuni paesi:
\begin{itemize}
\item imposte sulle bevande zuccherate e sui cibi contenenti grassi.
\end{itemize}
\item Assimilabile alla «sin tax» è anche l'imposizione sui giochi (lotterie,
scommesse, \emph{slot machine}, giochi d'azzardo on-line)
\end{itemize}
\end{frame}

%%%%%%%%%%%%%%%%%%%%%%%%%%%%%%%%%%%%%%%%%%%%
\begin{frame}{Imposte speciali: il gettito}
\begin{center}
\includegraphics[width=\textwidth]{./figure/gettito-imposte-speciali-Italia-2021.png}
\end{center}
\end{frame}


\section{Il coordinamento internazionale}

%%%%%%%%%%%%%%%%%%%%%%%%%%%%%%%%%%%%%%%%%%%%
\begin{frame} le imposte non dovrebbero distorcere
i flussi commerciali e la concorrenza.
\begin{description}
\item[{Ripartizione del gettito tra paesi:}] problemi di doppia
imposizione, nel paese di origine e di destinazione della merce.
\end{description}

Al fine di evitare il problema della doppia tassazione, è possibile
in linea di principio applicare uno dei seguenti due principi:
\begin{description}
\item[{Principio di origine:}] il bene è assoggettato ad Iva nel paese in cui è
prodotto (paese esportatore), e non subisce tassazione al passaggio della
frontiera
\item[{Principio di destinazione:}] l'imposta è applicata dal paese di
destinazione, ove il bene è consumato.  All'esportazione il bene viene
dunque "depurato" dell'Iva pagata fino a quel momento, mentre viene
applicata l'Iva ai beni importati.  È il principio applicato in Europa.
\end{description}
\end{frame}

%%%%%%%%%%%%%%%%%%%%%%%%%%%%%%%%%%%%%%%%%%%%
\begin{frame}
\centering
\begin{center}
\includegraphics[width=.9\textwidth]{./figure/IVA-destinazione.pdf}
\end{center}

\begin{itemize}
\item L'Iva è versata interamente nel paese Beta, e il bene è gravato dalla
relativa aliquota.
\end{itemize}
\end{frame}

%%%%%%%%%%%%%%%%%%%%%%%%%%%%%%%%%%%%%%%%%%%%
\begin{frame}
\begin{center}
\includegraphics[width=.9\textwidth]{./figure/IVA-origine.pdf}
\end{center}

\begin{itemize}
\item Il bene consumato in Beta è gravato dall'Iva di Beta, ma il gettito è
ripartito tra i diversi paesi in proporzione al valore aggiunto
\end{itemize}
\end{frame}

%%%%%%%%%%%%%%%%%%%%%%%%%%%%%%%%%%%%%%%%%%%%
\begin{frame}
\begin{center}
\includegraphics[width=.9\textwidth]{./figure/IVA-intracomunitaria.pdf}
\end{center}

\footnotesize
L'impresa importatrice applica il \emph{reverse charge}: l'IVA sulle importazioni
non viene versata alla dogana (che non c'è!); l'impresa importatrice registra
IVA sul bene importato sia a debito che a credito (con aliquota del paese
destinazione); al momento della vendita del bene viene versata l'IVA con
l'aliquota del paese destinazione.
\end{frame}


%%%%%%%%%%%%%%%%%%%%%%%%%%%%%%%%%%%%%%%%%%%%
\begin{frame}
\begin{center}
\includegraphics[width=.9\textwidth]{./figure/IVA-proposta-2018.pdf}
\end{center}
\end{frame}


\end{document}